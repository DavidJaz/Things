% A Modal Logic for Passing Constraints

\section{Introduction}

\begin{itemize}
    \item We wanted things.
    \item Noticed they "coordinate constraints".
    \item Need notion of part and passing constraints. [1 paragraph to here]
    \item We have a novel notion of "pre-part" as surjection out of behavior
    \item Inter-modalities: two pre-parts. Generalizes possibility and necessity.
\end{itemize}

\section{Systems and Their Parts: Behaviorwise}

%%%% Behavior Type of a system
When we constrain a part of a system, we are constraining \emph{what it does}. This suggests that we should model a system by its \emph{type of possible behaviors}, its \textbf{behavior type}.

Luckily, we won't need to settle on what precisely a behavior type is, so long as we can reason about it logically. For this, we need the behavior type of our system and its parts to be objects in a topos; then, we can use the internal logic of the topos to reason about our behavior types.

[This allows us the freedom to have sets varying in time etc etc.]

\begin{defn}
A \emph{behavior type} is an object of a topos $\Ba$ of behavior types.
\end{defn}

\begin{ex}
We will see a few running examples of systems considered through their behavior types in this paper. Let's introduce them now.

\begin{itemize}
    \item Consider a bicycle. The bicycle pedals might be moving at some speed $p$, and the bicycle wheels might be moving at some speed $w$, both real numbers. If the pedal is pushing at a certain speed, then the wheels are moving at a least a constant multiple of that speed. Therefore, we will take the behavior type of our bicycle to be
    $$B_{\type{Bicycle}} := \{(p, w) : \Rb \times \Rb \mid w \geq rp \}$$
    for some fixed $r : \Rb$.
    
    \item Consider a glass of water placed in a room of temperature $R$. The glass of water has temperature $T_t : \Rb$ for every time $t : \Nb$. By Newton's principle, the temperature of the water satisfies the following simple recurrence relation:
    $$T_{t + 1} = T_t + k(R - T_t).$$
    Therefore, the behavior type of this glass of water is
    $$B_{\type{Water}} := \{ T : \Nb \to \Rb \mid T_{t + 1} = T_t + k(R - T_t)\}.$$
    
    \item Consider an ecosystem consisting of foxes and rabbits. At any given time $t : \Nb$, there are $f_t$ foxes and $r_t$ rabbits, where we mask our uncertainty about the precise population by allowing these to be arbitrary real values. The population of the species at time $t + 1$ is determined by their population at time $t$ by the following standard recurrences $R_t(f, r)$:
    \begin{align*}
        f_{t + 1} &= (1 - d_f)f_t + c_f r_t f_t \\
        r_{t + 1} &= (1 + b_r)r_t - c_r r_t f_t.
    \end{align*}
    Therefore, we take the behavior type of this ecosystem to be
    $$B_{\type{Eco}} := \{(f, r) : \Nb \to \Rb \times \Rb \mid \forall t.\, R_t(f, r)\}.$$
    
    \item In algebraic geometry, we may think of the polynomial functions on a space as its possible behaviors. Thinking this way, the polynomial ring $\Rb[x, y]$ is the behavior type of the plane $\Rb \times \Rb$:
    $$B_{\Rb^2} := \Rb[x, y].$$
\end{itemize}
\end{ex}

%%%%%% Parts of a system as quotients of behavior type

If we know how the whole system $S$ is behaving like $b$, then we also know how any part $P$ of $S$ is behaving; we just look at what $P$ is doing while $S$ does $b$. In other words, there should be a function $|_P : B_S \to B_P$ from the behavior type of the whole system to the behavior type of the part of the system which restricts a behavior to that part. 

Moreover, every behavior of a part $P$ will arise from \emph{some} behavior of the whole system; how could a part of the system do something if the system as a whole had no behaviors in which $P$ was doing that thing. Remember, we are consider the part $P$ \emph{as a part of the system $S$}, not on its own. If we remove $P$ from the system $S$, it may be able to behave differently. But as a part of the system $S$, all the behaviors of $P$ are just things that $P$ is doing while the whole system behaves in a certain way.

Therefore, we come to the following definition.
\begin{defn}
The behavior type of a \emph{part} of a system $S$ is an epimorphism $|_P : B_S \to B_P$ out of $B_S$. The category of parts of $B_S$ has as objects the parts of $S$ and as morphisms the commuting triangles
        \begin{center}
            \begin{tikzcd}[ampersand replacement=\&]
             \& B_S \arrow[ld, "|_P"', two heads] \arrow[rd, "|_Q", two heads] \&  \\
            B_P \arrow[rr, "|_Q"'] \&  \& B_Q
            \end{tikzcd}
        \end{center}
Such a map $B_P \to B_Q$ is unique if it exists, in which case we write $P \geq Q$ and say that $Q$ is a part of $P$.
\end{defn}

In practice, we will have certain parts of $S$ in mind, and so will not consider the full lattice of quotients of $B_S$. 

\begin{rmk}
This definition should look a little backwards. Usually a ``part'' is a subobject; here we have defined a part to be a \emph{quotient}. What we have defined to be a part is often called a \emph{partition} (of $B_S$). 

What is happening here is that we are not considering the system $S$ itself, but rather its type of behaviors $B_S$. Often, the behaviors $B_S$ of a system $S$ may be realized as functions on some object; this gives us a contravariance in $S$, which we see in the definition of part. For comparison, consider the situation in algebraic geometry where one has a contravariant equivalence between the categories of algebraic varieties and the category of reduced finitely presented algebras; the algebras are algebras of polynomials functions on the varieties, which tell us the ``possible behaviors'' of these varieties, relative to the base field.
\end{rmk}

\begin{ex}
\begin{itemize}
    \item The two parts of the bicycle under consideration are the pedal and the wheel. Explicitly, the behavior types of these parts of the bicycle are the types of all possible behaviors which arise as some behavior of the whole bicycle:
    \begin{align*}
        B_{\type{Pedal}} &:= \{ p : \Rb \mid \exists w. (p, w) \in B_{\type{Bicycle}}\},\\
        B_{\type{Wheel}} &:= \{ w : \Rb \mid \exists p. (p, w) \in B_{\type{Bicycle}}\}.
    \end{align*}
    In this case, every real number is a possible speed of the pedal, and every real number a possible speed of the wheel.
    
    \item In the system $\type{Water}$ of the cup of water sitting in the room, there is just one thing we are considering the behavior of: the cup. But, we can see this behavior at many different times. For every time $t : \Nb$, we get a part $\type{Water}_t$ of the cup at time $t$ with behaviors
    $$B_{\type{Water}_i} := \{ x : \Rb \mid \exists T \in B_{\type{Water}}.\, T_t = x\}.$$
    In fact, for any set $D \subseteq \Nb$ of times, we get the behavior type of the cup during $D$:
    $$B_{\type{Water}_D} := \{ x : D \to \Rb \mid \exists T \in B_{\type{Water}}.\, \forall d : D.\, T_d = x_d\}.$$
    
    \item The ecosystem consisting of foxes and rabbits is more complicated than the cup, but the principle is the same. We can consider the system at different times, and restrict our attention to just foxes or rabbits as we please. In particular, we let $\type{Fox}_t$ be the system of foxes at time $t$, and $\type{Rabbit}_t$ be the system of rabbits at time $t$. These have behavior types
    \begin{align*}
        B_{\type{Fox}_t} &:= \{ x : \Rb \mid \exists (f, r) \in B_{\type{Eco}}.\, f_t = x\},\\
        B_{\type{Rabbit}_t} &:= \{ x : \Rb \mid \exists (f, r) \in B_{\type{Eco}}.\, r_t = x\}.
    \end{align*}
    
    \item In algebraic geometry, the parts of the plane $\Rb^2$ under consideration are the algebraic subsets. These are subsets carved out by algebraic equations. If $f : \Rb[x, y]$ is a polynomial, then $\Rb[x, y]/(f)$ is the algebra of polynomial functions where $f$ is $0$. This is the algebra of polynomial functions on the subspace $\{(x, y) : \Rb^2 \mid f(x, y) = 0\}$ of the plane. In the case that $f = x^2 + y^2 - 1$, then this subspace is the circle, and so we see that the behavior type of the circle is
    $$B_{\type{Circle}} := \Rb[x, y]/(x^2 + y^2 - 1).$$
    
\end{itemize}
\end{ex}

An epimorphism out of an object (in a topos) may equally be presented by its kernel pair, which is an equivalence relation on that object. In this case, we may view the equivalence relation $\sim_P$ on $S$ induced by $S \twoheadrightarrow P$ as \emph{observational equivalence}; in other words, two behaviors $s, s' : S$ of the whole system are \emph{observationally equivalent} relative to $P$, $s \sim_P s'$, if and only if they restrict to these same behavior of $P$. This is clearest when thinking of $P$ as some measuring device in a larger system; two behaviors of the whole system are observationally equivalent relative to our measuring device when it measures them to be the same.

\subsection{Compatibility, Determination, and the Lattice of Parts}

Now, we turn our attention to how the behaviors of various part of the system relate. The most basic relation that behaviors of two different parts can have is that they are mutually realizable by a behavior of the whole system. We call this relation \emph{compatibility}.

\begin{defn}
If $P$ and $Q$ are parts of $S$, then we say behaviors $a : B_P$ and $b : B_Q$ are \emph{compatible} if there is a behavior $s : B_S$ of the whole system that restricts to both $a$ and $b$.
$$\mathfrak{c}(a, b) :\equiv \exists s : S.\, a = s|_P\, \wedge\, s|_Q = b.$$

Generally, if $a_i : P_i$ is some family of behaviors, then this family is said to be \emph{compatible} if there is an $s : S$ such that $s|_{P_i} = a_i$ for all $i$.
\end{defn}

In other words, two behaviors of two parts are compatible if there is a behavior of the whole system that restricts to both of them. 

\begin{ex}
\begin{itemize}
\item In the bicycle example, we see that a speed $p$ of the pedal is compatible with a speed $w$ of the wheel if and only if $w \geq rp$:
$$\mathfrak{c}(p, w) = w \geq rp.$$

\item In the cup of water example, a temperature $T^0 : B_{\type{Water}_t}$ at time $t$ is compatible with a temperature $T^1 : B_{\type{Water}_{t'}}$ at a later time $t'$ are compatible if and only if $T^1$ follows from $T^0$ via the recurrence relation. In particular, if $t' = t + 1$, then
$$\mathfrak{c}(T^0, T^1) = (T^1 = T^0 + k(R - T^0)).$$

\item In the ecosystem example, we have a number a different comparisons to choose from. A fox population $f^0 : B_{\type{Fox}_t}$ at time $t$ is compatible with $f^1 : B_{\type{Fox}_{t + 1}}$ at time $t + 1$ if and only if there is simultaneous rabbit population $r^0$ so that $f^1 = (1 - d_f)f^0 + c_f r^0 f^0$. 

Two simultaneous fox and rabbit populations are compatible if and only if there is some history of the ecosystem which achieves those population at that time. In particular, any two populations of foxes and rabbits at time $0$ are compatible.

\item In the algebraic geometry example, a part is an algebraic subset of the plane and its behaviors are the polynomial functions defined their. Two behaviors are compatible if they extend to extend to the same function of the plane. For example, a behavior $p$ of the part $\{(a, b)\} \subseteq \Rb^2$ is just a real number; for a behavior $q : B_X$ of some part $X$ to be compatible with $b$ means that $Q(a, b) = p$ for some polynomial function $Q$ on the plane which restricts to $q$ on $X$.
\end{itemize}
\end{ex}

In fact, we can now express some of the algebra of parts in terms of compatibility.

\begin{claim}
The meet $P \wedge Q$ of two parts $P$ and $Q$ of $S$ have behaviors given by the following pushout.
\[
\begin{tikzcd}
 & B_S \arrow[ld, two heads] \arrow[rd, two heads] &  \\
B_P \arrow[rd, two heads] &  & B_Q \arrow[ld, two heads] \\
 & B_{P \wedge Q} & 
\end{tikzcd}
\]
In other words, a behavior of $P \cap Q$ is either a behavior of $P$ or a behavior of $Q$, where these are considered equal if they are compatible.
$$B_{P \cap Q} \cong \frac{B_P + B_Q}{\mathfrak{c}}.$$

Dually, the join $P \cup Q$ has bevahaviors given by the image factorization of the induced map $B_S \to B_P \times B_Q$. In other words, a behavior of $P \vee Q$ is a pair of compatible behaviors from $P$ and from $Q$.
$$B_{P \cup Q} \cong \{(a, b) : B_P \times B_Q | \mathfrak{c}(a, b)\}.$$

Furthermore, the largest part $\top$ is $S$, and the smallest part $\bot$ is the \emph{support} (or \emph{duration}, thinking timewise) of $S$, the image factorization of the terminal map.
\end{claim}

Therefore, when we say that any behavior of $P$ is compatible with any behavior of $Q$, we are saying that they are disjoint as parts, i.e. that their meet is empty.
$$\forall a : B_P.\, \forall b : B_Q.\, \mathfrak{c}(a, b) \quad\Rightarrow\quad B_{P \cap Q} = \bot$$

In general, we will be more interested in joins than in meets because joins are easier to work with (being subobjects of a product, rather than quotients of a disjoint union by a non-transitive relation).
\begin{ex}
\begin{itemize}
    \item In the example of the bicycle, note that we have
    $$B_{\type{Bicycle}} = B_{\type{Pedal} \cup \type{Wheel}},$$
    since a behavior of the bicycle was defined precisely to be a behavior of a pedal and a wheel satisfying a compatibility constraint.
    
    \item In the example of the cup of water, the behaviors $B_{\type{Cup}_D}$ over a duration $D \subset \Nb$ of times are the union of the behaviors $B_{\type{Cup}_d}$ for each time $d \in D$:
    $$B_{\type{Cup}_D} = B_{\bigcup \type{Cup}_d}.$$
    
    \item Similarly, in the ecosystem example, the parts of the ecosystem at various times are the join of parts at particular times. More interestingly, recall that every behavior of $\type{Fox}_0$ (starting population of foxes) is compatible with every behavior of $\type{Rabbit}_0$ (starting population of rabbits). Therefore, 
    $$B_{\type{Fox}_0 \cap \type{Rabbit}_0} = \bot$$
    
    \item Consider two algebraic subsets $X$ and $Y$ of the plane and let $I$ and $J$ be the ideals of polynomials which vanish on $X$ and $Y$ respectively. Then we know that $B_X = \Rb[x, y]/I$ and $B_Y = \Rb[x, y]/J$. The join has behaviors 
    \begin{align*}
        B_{X \cup Y} &= \{(f, g) : \Rb[x, y]/I \times \Rb[x, y]/J \mid \exists h : \Rb[x, y].\, h + I = f + I \mbox{ and } h + J = g + J\} \\
            &= \{(h + I, h + J) \mid h : \Rb[x, y]\}.
    \end{align*}
    We note that $B_{X \cup Y}$ is a ring with operations taken componentwise and that the restriction map $B_{\Rb^2} \to B_{X \cup Y}$ is a homomorphism. The kernel is the set of those $h : \Rb[x, y]$ for which $h + I = I$ and $h + J = J$, i.e. precisely those $h \in I \cap J$. Therefore, 
    $$B_{X \cup Y} = \Rb[x, y]/(I \cap J),$$
    which represents the union in the usual geometric sense.  
    
\end{itemize}
\end{ex}

We might say that a part $P$ \emph{determines} a part $Q$ if for every behavior $a : B_P$, there is a unique compatible behavior $b : B_Q$. This is a much stronger notion than compatibility, and it can be used to the mereology of parts.

\begin{defn}
If $P$ and $Q$ are parts of $S$, and $a : B_P$ and $b : B_Q$, then $a$ \emph{determines} $b$ if every behavior of the whole system $s$ which restricts to $a$ also restricts to $b$.
$$\mathfrak{d}(a, b) :\equiv \forall s : S.\, s|_P = a \Rightarrow s|_Q = b.$$
\end{defn}


\begin{lemma}
A behavior always determines uniquely: if $\mathfrak{d}(a, b)$ and $\mathfrak{d}(a, b')$, then $b = b'$.
\end{lemma}
\begin{proof}
We know there is some $s : S$ which restricts to $a$. Since $a$ determines $b$ and $b'$, $s$ restricts to both $b$ and $b'$; but then $b = b'$.
\end{proof}


\begin{claim}\label{Things:lem:Determines.Mereology}
For parts $P$ and $Q$ of $S$, the following are equivalent:
\begin{enumerate}
    \item $Q$ is a part of $P$.
    \item $P$ determines $Q$, in the sense that $\forall a : B_P.\, \exists! b : B_Q.\, \mathfrak{c}(a, b)$.
    \item $P$ determines $Q$, in the sense that for every $a : B_P$ there is a $b : B_Q$ such that $a$ determines $b$. In other words, $\forall a : B_P.\, \exists b : B_Q.\, \mathfrak{d}(a, b)$.
    \item For all $p : B_P$ and $q : B_Q$, if $p$ is compatible with $q$, then $p$ determines $q$.
\end{enumerate}
\end{claim}
\begin{proof}
\begin{itemize}
    \item $1 \Rightarrow 2:$ Suppose we have $f : B_P \twoheadrightarrow B_Q$ under $B_S$. Then for $a : B_P$, we have $f(a) : B_Q$ and if $s : B_S$ restricts to $a$, then $s|_B = f(s|_A) = f(a)$ so that $a$ is compatible with $f(a)$. If $b : B_Q$ is also compatible with $a$, then let $s : B_S$ restrict to $a$; then $s$ restricts to $b$ but also to $f(a)$, so that $b = f(a)$.
    \item $2 \Rightarrow 3:$ Suppose $B_P$ determines $B_Q$. Then for $a : B_P$, let $b : B_Q$ be the guaranteed unique compatible behavior of $Q$. If $s : B_S$ restricts to $a$, then $a$ is compatible with $s|_Q$ and so $s|_Q = b$; therefore, $a$ determines $b$.
    \item $3 \Rightarrow 1:$ By the lemma, $a$ determines $b$ uniquely; thus we get a function $f : B_P \to B_Q$ sending $a$ to $f(a) := b$. Now, for any $b : B_Q$, there is some $s : B_S$ restricting to it. By hypothesis, $s|_P$ determines some $f(a) : B_Q$; but then $s|_Q = f(a)$ so that $b = f(a)$ and $f$ is epi. Finally, if $s : B_S$, then $f(s|_P) = s|_Q$, so that $Q$ is part of $P$.
    \item $3 \iff 4:$ Since for every $p$, there is a $q$ compatible with it, $4$ implies $3$. On the other hand, assuming $3$ and that $p$ is compatible with $q$, we know by $2$ that $q$ is the unique behavior of $Q$ compatible with $p$, so that $p$ determines $q$.
\end{itemize}
\end{proof}

\begin{ex}
\begin{itemize}
    \item In the example of the bicycle, neither part determines the other.
    \item In the example of the cup of water, each temperature at time $t$ determines the temperature at time $t + 1$ by the recurrence relation. Therefore, we have $\type{Cup}_t \geq \type{Cup}_{t + 1}$ by Lemma \ref{Things:lem:Determines.Mereology}; in other words, a cup at time $t + 1$ is ``part'' of a cup at time $t$. This may seem odd, but remember that our notion of part was operational; in this determinisitic system, the behaviors type of temperatures at time $t$ \emph{contains} all the information of the behaviors at later times.
    \item As a determinisitic dynamical system, the same dicussion works for the ecosystem example as for the cup of water example.
    \item algebra
\end{itemize}
\end{ex}


\section{Constraints, Compatibility, and Ensurance}

We will identify a constraint $\phi$ on a part $P$ with the predicate ``satisfies $\phi$'' on behaviors $B_P$ of $P$. In other words, we have the following definition.
\begin{defn}
A \emph{constraint} on a part $P$ is a map $\phi : B_P \to \Prop$. The type of constraints on $P$ is $\Prop^P$.
\end{defn}

Since behavior types form a topos, for parts $Q \leq P$, we get an adjoint triple:
    \begin{center}
        \begin{tikzcd}[ampersand replacement=\&]
        \Prop^{B_P} \arrow[rr, "\exists^P_Q", bend left] \arrow[rr, "\forall^P_Q"', bend right] \&  \& \Prop^{B_Q} \arrow[ll, "\Delta^P_Q" description]
        \end{tikzcd}
    \end{center}\pause 
These are defined logically as follows:
\begin{align*}
    \exists^P_Q \phi(q) &:= \exists p : B_P.\, p|_Q = q \meet \phi(p) \\
    \Delta^P_Q \psi(p) &:= \psi(p|_Q) \\
    \forall^P_Q \phi(q) &:= \forall p : B_P.\, p|_Q = q \Rightarrow \phi(p)
\end{align*}
We will write $\exists_P$, $\Delta_P$, and $\forall_P$ for $\exists^S_P$, $\Delta^S_P$ and $\forall^S_P$ respectively. We also note that these operations are functorial, meaning that for $R \leq Q \leq P$,
\begin{align*}
    \exists^P_Q \exists^Q_R &= \exists^P_R, \\
    \Delta^P_Q \Delta^Q_R &= \Delta^P_R, \\
    \forall^P_Q \forall^Q_R &= \forall^P_R.
\end{align*}

\begin{claim}\label{Thing:lem:Global.Modality.Props}
For part $P$ of system $S$, write $s \sim_P s'$ for the relation $s|_P = s'|_P$. Then:
\begin{enumerate}
    \item $$\Delta_P \exists_P \phi(s) = \exists s'.\, s\sim_P s' \wedge \phi(s')$$
    \item $$\Delta_P \forall_P \phi(s) = \forall s'.\, s \sim_P s' \Rightarrow \phi(s')$$
    \item \begin{align*}
        \phi &\vdash \Delta_P \exists_P \phi\\
        \Delta_P \forall_P \phi &\vdash \phi 
    \end{align*}
\end{enumerate}
\end{claim}

Now we turn to the question of how constraints on the behavior of some part of the system contrain the behavior of other parts. 

     \begin{defn}
        A constraint $\phi$ on a part $P$ induces a constraint on a part $Q$ in two universal ways:
        \begin{itemize}
            \item ``Is compatible with $\phi$'': $\compat{P}{Q} :\equiv \exists_Q \Delta_P$
                $$\compat{P}{Q} \phi(q) :\equiv  \exists s : B_S.\, s|_Q = q \wedge \phi(s|_P) = \exists p : B_P.\, \mathfrak{c}(p, q) \meet \phi(p).$$
            \item ``Ensures $\phi$'': $\ensure{P}{Q} :\equiv \forall_Q  \Delta_P$
                $$\ensure{P}{Q} \phi(q) :\equiv \forall s : B_S.\, s|_Q = q \Rightarrow \phi(s|_P) = \forall p : B_p.\, \mathfrak{c}(p, q) \Rightarrow \phi(p).$$
        \end{itemize} 
    \end{defn}
    
    A constraint $\phi$ on $P$ being compatible with a behavior $q$ of $Q$ means that $Q$ can be doing $Q$ while $P$ is satisfying $\phi$; we write this as $\compat{P}{Q}\phi(q)$. A behavior $q$ of $Q$ ensures $\phi$ on $P$ means that if $Q$ does $q$, then $P$ must satisfy $\phi$; we write this as $\ensure{P}{Q}\phi(q)$.
    
    These symbols are chosen due to their relation to the usual modalities of possibility ($\Diamond$) and necessity ($\Box$); a behavior $q$ is compatible with $\phi$ if it is \emph{possible} that $P$ satisfies $\phi$ while $Q$ does $q$, and a behavior $q$ ensure $\phi$ if it is \emph{necessary} that $P$ satisfies $\phi$ while $Q$ does $q$. Indeed, we will be able to recover the usual possibility and necessity modalities from our compatiblity and ensurance operators.
    
    Note that compatibility and determination appear as particular cases of the compatibility and ensurance operators.
    \begin{align*}
        \mathfrak{c}(p, q) &= \compat{P}{Q}(=p)(q) = \compat{Q}{P}(=q)(p) \\
        \mathfrak{d}(p, q) &= \ensure{Q}{P}(=q)(p)
    \end{align*}
    
    \begin{ex}
    \begin{itemize}
        \item In the example of the bicycle, if the wheels are moving at $2$ mph, then the pedal must be moving less than that. In other words, $\ensure{P}{W}(\leq 2)(2)$ holds.
        
        \item If the cup of water has temperature $T^0$ at time $0$, then it cannot have a temperature further away from the ambient room temperature $R$ at a later time. Therefore, 
        $$|R - (-)| > |R - T^0| \vdash \neg \compat{T_0}{T_t}(=T^0).$$
        
        \item Suppose that in the ecosystem example, one was given the goal of introducing a fox population at time $0$ in order to keep the rabbit population in check after a given deadline $d$. Let's say that being kept in check means being between two fixed bounds, $$r_t \mapsto \term{inCheck}(r_t) := k_1 < r_t < k_2$$
        so that $\term{inCheck} : B_{R_t} \to \Prop$ is a constraint on rabbits at time $t$. The constraint of being kept in check for all times after the deadline $d$ is the constraint
        $$r \mapsto \forall t \geq d.\, \term{inCheck}(r_t)$$
        on the join $\bigvee_{t \geq d} R_t$. The goal may then be expressed as finding a starting fox population $f_0$ which ensures that the rabbit population is kept in check at all times after the deadline:
        $$\ensure{\bigvee_{t \geq d} R_t}{F_0}(\forall t \geq d.\, \term{inCheck})(f_0).$$   
        \end{itemize}
 \end{ex}
 
    Though we are setting up our theory in the generality of an arbitrary topos, the topos of sets is a very important special case. The internal logic of this topos is Boolean, in that it satisfies the law of excluded middle. In this case, our operators are inter-definable by conjugating with negation.
    
    \begin{claim}
    Assuming Boolean logic, compatibility and ensurance are de Morgan duals. That is, $\neg \compat{P}{Q} \neg = \ensure{P}{Q}$.
    \end{claim}
    \begin{proof}
    \begin{align*}
        \neg \compat{P}{Q} \neg \phi(q) &= \neg \exists p.\, \mathfrak{c}(p, q) \meet \neg \phi(p) \\
            &= \forall p.\, \neg (\mathfrak{c}(p, q) \meet \neg \phi(p)) \\
            &= \forall p.\, \neg \mathfrak{c}(p, q) \join \neg\neg \phi(p) \\
            &= \forall p.\, \mathfrak{c}(p, q) \Rightarrow \phi(p).
    \end{align*}
    \end{proof}
    
\subsection{Properties of Compatibility and Ensurance}
We now develop the basic theory of the compatibility and ensurance operators.    
    
    \begin{claim} Let $P$ and $Q$ be parts of the system $S$. Then:
    \begin{enumerate}
        \item A constraint is compatible with itself if and only if it is satisfied if and only if it ensures itself. In other words, $\compat{P}{P} = \id = \ensure{P}{P}$.
        \item If a constraint $\phi$ entails $\psi$, then being compatible with $\phi$ entails being compatible with $\psi$, and ensuring $\phi$ entails ensuring $\phi$. That is, $\compat{P}{Q}$ and $\ensure{P}{Q}$ are monotone.
        \item If $q$ ensures that $P$ does $\phi$, then $q$ is compatible with $P$ doing $\phi$. That is, $\ensure{P}{Q} \vdash \compat{P}{Q}$
        \item Being compatible with $\phi$ entails $\psi$ if and only if $\phi$ entails ensuring $\psi$. That is, $\compat{P}{Q}$ is left adjoint to $\ensure{Q}{P}$.
        \item $\compat{P}{Q}$ commutes with $\join$ and $\exists$, and $\ensure{P}{Q}$ commutes with $\meet$ and $\forall$.
        
    \end{enumerate}
    \end{claim}
    \begin{proof}
    \begin{enumerate}
        \item As they are adjoint, it suffices to show that just one is the identity. Suppose $\ensure{P}{P}\phi(p)$; that is, for all $s : S$, $s|_P = p$ implies $\phi(p)$. Since $|_P$ is epi, there is some $s$ that restricts to $p$, and therefore $\phi(p)$.
        \item As the composite of monotone maps, both maps are monotone.
        \item Let $\phi : B_P \to \Prop$ be a constraint on $P$. Suppose that $\ensure{P}{Q}\phi(q)$ for $q : Q$, that is that for all $s : B_S$, if $s|_Q = q$, then $\phi(s|_P)$. Since $|_Q$ is epi, there is an $s$ such that $s|_Q = q$, and therefore $\phi(s|_P)$; so $\ensure{P}{Q}\phi(q) \vdash \compat{P}{Q}\phi(q)$.
        \item Follows from the adjunctions $\exists_Q \dashv \Delta_Q$ and $\Delta_P \dashv \forall_P$ as follows:
        \begin{align*}
            \compat{P}{Q} \phi &\vdash \psi \\ 
            \exists_Q \Delta_P \phi &\vdash \psi \\
            \Delta_P \phi &\vdash \Delta_Q \psi \\ 
            \phi &\vdash \forall_P \Delta_Q \psi \\
            \phi &\vdash \ensure{Q}{P} \psi.
        \end{align*}
        \item By the general properties of adjoints.
        
    \end{enumerate}
    \end{proof}
    
 To make the adjointness of $\compat{P}{Q}$ and $\ensure{Q}{P}$ more visceral, consider the following example: by putting my left hand on the wall, I can ensure my right hand is within ten feet of the wall. This is the same as saying that if the behavior of my right hand is compatible with my left hand being on the wall, then my right hand is within 10 feet of the way.
    
    The unit and counit of this adjunction are interesting as well. The unit says that whatever my left hand is doing, doing this ensures that my right hand is compatible with doing it. The counit says that my right hand being compatible with my left hand ensuring that my right hand is doing something implies that my right hand is doing that thing. For example, if my right hand is compatible with my left hand ensuring that it is within 10 feet of the wall, then my right hand must be within 10 feet of the wall.
    \begin{claim}
    Let $P$, $Q$, and $R$ be parts of system $S$. Then:
    \begin{enumerate}
        \item If a behavior of $R$ is compatible with a constraint on $P$, then it is compatible with a behavior of $Q$ being compatible with that constraint. That is, $\Diamond^P_R \vdash \Diamond^Q_R  \Diamond^P_Q$
        \item If a behavior of $R$ ensures that a behavior of $Q$ ensures some constraint on $P$, then that behavior of $R$ ensures that constraint on $P$. $\Box^Q_R\Box^P_Q \vdash \Box_R^P$
    \end{enumerate}
    \end{claim}
    \begin{proof}
    This follows immediately from Claim \ref{Thing:lem:Global.Modality.Props}, since 
    $$\compat{P}{R} = \exists_R \Delta_P \vdash \exists_R \Delta_Q \exists_Q \Delta_P = \compat{Q}{R}\compat{P}{Q}$$
    $$\ensure{Q}{R}\ensure{P}{Q} = \forall_R \Delta_Q \forall_Q \Delta_P \vdash \forall_R \Delta_P = \ensure{P}{R}$$
    \end{proof}
    
\begin{ex}
\begin{itemize}
    \item We can see a higher order ensurance in the ecosystem example. If there are any rabbits at time $0$, and if the rabbit population is bounded independent of time, then the rabbits must ensure that there are foxes, and that the foxes ensure there are rabbits:
    $$r_0 \geq 0 \meet r < k \vdash \ensure{F}{R} (f > 0 \meet \ensure{R}{F}(r > 0)).$$ 
    If there are no foxes, then the rabbit population is unbounded, and if there are foxes, then there must be rabbits for them to eat. We see that this ecosystem model exhibits a rudimentary form of symbiosis; though the foxes eat the rabbits, they counter-intuitively must ensure that the rabbits do not go extinct less they themselves go extinct.
\end{itemize}
\end{ex}
 
\begin{claim}
\begin{enumerate}
    \item $\Diamond^P_{Q \cap R}\phi = \exists q : Q,\, r : R.\, \mathfrak{c}(q,r) \wedge \Diamond^P_{Q \cup R}\phi(q, r)$.\bigskip
    \item $\Diamond^P_{Q \cup R} \phi(q, r) \Rightarrow \Diamond^P_Q \phi(q) \wedge \Diamond^P_R \phi(r)$.\bigskip
    \item $\Box^P_{Q \cap R}\phi = \forall q : Q,\, r : R.\, \mathfrak{c}(q,r) \Rightarrow \Box^P_{Q \cup R}\phi(q, r)$.\bigskip
    \item $\Box^P_Q \phi(q) \vee \Box^P_R \phi(r)  \Rightarrow  \Box^P_{Q \cup R} \phi(q, r)$.
\end{enumerate}
\end{claim}
\begin{proof}
We prove the first two as the last two are dual.
\begin{enumerate}
    \item $\Rightarrow:$ Suppose that $\compat{P}{Q \cap R}\phi(a)$. Then there is an $s : B_S$ so that $s|_{Q \cap R} = a$ and $\phi(s|_P)$. But then $s|_Q$ and $s|_R$ are compatible and $s|_{Q \cup R} = (s|_Q,\, s|_R)$, so that $\compat{P}{Q \cup R}\phi(s|_Q,\, s|_R)$.
    
    $\Leftarrow:$ Suppose that there are compatible $q : Q$ and $r : R$ with $\compat{P}{Q \cup R}\phi(q, r)$. Since $q$ and $r$ are compatible, $q|_{Q \cap R} = r|_{Q \cap R}$, so that $\compat{P}{Q \cap R}\phi(q|_{Q \cap R})$.
    \item 
        \begin{align*}
            \compat{P}{Q \cap R}\phi(q, r) &= \exists s.\, s|_{Q \cup R} = (q, r) \meet \phi(s|_P) \\
                &= \exists s.\, s|_Q = q \meet s|_R = r \meet \phi(s|_P) \\
                &\Rightarrow (\exists s.\, s|_Q = q \meet \phi(s|_P)) \meet (\exists s.\, s|_R = r \meet \phi(s|_P)) \\
                &= \compat{P}{Q}\phi(q) \wedge \compat{P}{R}\phi(r).
        \end{align*}
\end{enumerate}
\end{proof}

\begin{claim}
Suppose that $P \geq Q$. Then
\begin{align*}
    \compat{P}{Q} &= \exists^P_Q = \exists p : B_P.\, p|_Q = q \meet \phi(p) \\
    \ensure{P}{Q} &= \forall^P_Q = \forall p : B_P.\, p|_Q = q \Rightarrow \phi(p)
    \compat{Q}{P} = \Delta^P_Q = \ensure{Q}{P}
\end{align*}
Furthermore, if $\compat{Q}{P} \vdash \ensure{Q}{P}$, then $P \geq Q$.
\end{claim}
\begin{proof}
\begin{align*}
    \compat{P}{Q} &= \exists_Q \Delta_P \\
        &= \exists^P_Q \exists_P \Delta_P \\
        &= \exists^P_Q
\end{align*}
and dually.

Finally, if $\compat{Q}{P} \vdash \ensure{Q}{P}$, then in particular $\compat{Q}{P}(=q)(p) \vdash \ensure{Q}{P}(=q)(p)$, or in other words $\mathfrak{c}(p, q) \vdash \mathfrak{d}(p, q)$; but this was one of the equivalent conditions of Lemma \ref{Things:lem:Determines.Mereology}.
\end{proof}

\begin{claim}
Suppose that $P \leq P'$ and $Q' \leq Q$. Then
\begin{enumerate}
    \item $\compat{P'}{P} \compat{Q}{P'} = \compat{Q}{P}$.
    \item $\ensure{P'}{P} \ensure{Q}{P'} = \ensure{Q}{P}$.
    \item $\compat{P}{Q} \Delta^{Q}_{Q'} = \compat{P}{Q'}$.
    \item $\ensure{P}{Q} \Delta^{Q}_{Q'} = \ensure{P}{Q'}$.
\end{enumerate}
\end{claim}
\begin{proof}
By the functoriality of $\exists$, $\forall$, and $\Delta$.
\end{proof}

\begin{ex}
In the ecosystem example, we were trying to ensure that rabbits were kept in check after the deadline $d$:
$$\ensure{\bigvee_{t \geq d}}{F_0}(\forall t \geq d.\, \term{inCheck}).$$
Using the above lemmas, we can change this into a simpler form.
\begin{align*}
    \ensure{\bigvee_{t \geq d} R_t}{F_0}(\forall t \geq d.\, \term{inCheck}(r_t)) &= \forall t \geq d.\, \ensure{\bigvee_{t \geq d} R_t}{F_0}\term{inCheck}(r_t) &\mbox{since ensurance is a right adjoint.} \\
    &= \forall t \geq d.\, \ensure{R_t}{F_0}\term{inCheck}(r_t) &\mbox{because $\term{inCheck}(r_t)$ is, more explictly, $\Delta^{\bigvee_{t \geq d} R_t}_{R_t}\term{inCheck}$.}
\end{align*}
\end{ex}

\begin{claim}
The operators $\compat{Q}{P}\compat{P}{Q}$ and $\ensure{Q}{P}\ensure{P}{Q}$ are a pair of adjoint modalities on $\Prop^{B_P}$. They are the identity modality if and only if $P \leq Q$.
\end{claim}
\begin{proof}
We can see that they are adjoint as follows:
\begin{align*}
    \compat{Q}{P}\compat{P}{Q} \phi &\vdash \psi \\
    \compat{P}{Q} \phi &\vdash \ensure{P}{Q} \psi \\
    \phi &\vdash \ensure{Q}{P} \ensure{P}{Q} \psi
\end{align*}
Now, $\compat{Q}{P}\compat{P}{Q}$ is the identity if and only if $\compat{Q}{P}\compat{P}{Q} \phi &\vdash \phi$ for all $\phi$, which occurs if and only if $\compat{P}{Q}\phi \vdash \ensure{P}{Q}\phi$ for all $\phi$, which occurs if and only if $P \leq Q$.
\end{proof}

Assuming that the behavior type $B_P$ of $P$ is well-supported (in that the terminal map $! : B_P \to \ast$ is epi), then the adjoint modalities
$$\compat{\bot}{P}\compat{P}{\bot} \dashv \ensure{\bot}{P}\ensure{P}{\bot}$$
are the usual possibility and necessity modalities on $B_P$. Then interpretation is that a behavior is compatible with nothing if and only if it is possible, and that it is ensured by nothing if and only if it is necessary.

\section{Example: how behavioral constraints are passed through time}

A temporal behavior type is a sheaf $S$ on the interval domain $IR$. 

This is a system that we can analyze, but because our audience may not be experts in topos theory, we will simplify significantly in several steps and then give a self-contained point of view.

We will show that we can pass constraints between the past and the future using the compatible and ensure modalities. 